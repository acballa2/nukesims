\documentclass[11pt]{article}
%ok hey a cool thing is you can comment out code in here that's nice
%Overleaf is a website with templates and stuff that you can use to skip a lot of this formatting
%This top part before \begin{document} is called the preamble
%Let's do some crazy formatting stuff
\usepackage[margin=1in]{geometry}
\usepackage{fancyhdr}
%don't know why it's called fancy but it lets you not have default header and footer
\pagestyle{fancy}
%don't know what this does but I assume it sets the scene for us to use fancy next
\fancyhead{}
\fancyfoot{}
%doing that clears the default settings for the header and footer
\fancyhead[L]{\slshape\MakeUppercase{Maybe a Title}}
\fancyhead[R]{Anna Balla}
%The L or the R is for left or right, slshape means italics, MakeUppercase (make sure M and U are uppercase) makes it uppercase
\fancyfoot[R]{\thepage}
%gives you a page number
\setlength{\headheight}{14pt}
%I don't know exactly what just happened but this is what you do when it tells you your header is too small
%\renewcommand{\headrulewidth}{0pt} gets rid of the line in the header if you so choose

\begin{document}
%\title{Practice \LaTeX \ Document}
%\author{Anna Balla}
%\date{\today}
%\maketitle
%I commented all that out because the page that has that title on it doesn't seem to want to follow my formatting stuff so I'm gonna see if there's a better way
\begin{titlepage}
\begin{center}
\vspace*{1cm}
%pushes the stuff below it down whatever your specified amount is (can be in cm or in)
\Large{\textbf{Here's maybe the name of the thing}}\\
\vfill
%adds vertical space all the way to the end of the page
\line(1,0){400}\\[1mm]
%horizontal line
\Huge{\textbf{I can't find the ARFC report template}}\\[3mm]
\Large{\textbf{So this is what we're using for now}}\\[1mm]
%these little bracketed measuremens add space between lines
\line(1,0){400}\\
\vfill
%doing this again sort of centers the title vertically though I have pretty much no idea what you'd do if you wanted to move it around any more
\large{By Anna Balla}
\end{center}
\end{titlepage}

%YO AND THE PAGE NUMBERS DONT APPEAR ON THE TITLE PAGE FU YEAH
\pagebreak

\tableofcontents
%\tableofcontents will take all your sections from down under and turn them into a table of contents for you, that's fantastic
\thispagestyle{empty}
\clearpage
%these things make it so there's no header and footer and page number on your table of contents

\setcounter{page}{1}
%makes it so page 1 is numbered 1 instead of 2

%here's some stuff about formatting your line spacing etc. & you can put this all the way before the title page if you want
%\parindent 0ex %gets rid of any indents
%\setlength{\parindent}{4em} %sets how far in your indent is
%\setlength{\parskip}{1em} %sets how big the space is after a paragraph
%\renewcommand{\baselinestretch}{1.5} %sets the line spacing

This is a line of text.\\ 
This is an equation $y=mx+b$.\\
I can also \textbf{bold things} and \textit{italicize things}.\\
I can also put equations in equation format like so: $$y=ax^2+bx+c$$
Double backslash is a soft return and if you actually hit return twice you can create a new paragraph with an indent.

Here's the new paragraph.\\
You can also make words \begin{Large} bigger \end{Large} or \begin{small} smaller \end{small}. Other options are \begin{large} large \end{large} or \begin{Huge} Huge \end{Huge} or \begin{tiny} tiny \end{tiny}. You get the idea.

\begin{center} This is how you would center your text. \end{center}

\begin{flushleft}This is for left-justified text.\end{flushleft}

\begin{flushright} This is for right-justified text.\end{flushright}

Also Latex seems to not like the soft return if you are switching up some formatting. 

Lets make sections\footnote{Wild! you can make footnotes!}

\section{Section 1}
\subsection{Subsection 1}
Here's some info in section 1 yee-haw.
\subsection{Subsection 2}
\section{Section 2}

Look they even make the text nice and bold and big for ya.  Could be helpful for making a table of contents. Maybe press f1 to compile quick. Ok actually you gotta press Fn f1 because your f1 is a mute button. 
\end{document}